%%%%%%%%%%%%%%%%%%%%%%%%%%%%%%%%%%%%%%%%%
% Important note:
% This template requires the resume.cls file to be in the same directory as the
% .tex file. The resume.cls file provides the resume style used for structuring the
% document.
%
%%%%%%%%%%%%%%%%%%%%%%%%%%%%%%%%%%%%%%%%%

%----------------------------------------------------------------------------------------
%	PACKAGES AND OTHER DOCUMENT CONFIGURATIONS
%----------------------------------------------------------------------------------------

\documentclass{resume} % Use the custom resume.cls style

\usepackage[left=0.75in,top=0.6in,right=0.75in,bottom=0.6in]{geometry} % Document margins
\usepackage{xcolor}

\usepackage{fancyhdr}
\pagestyle{fancy}
\fancyhf{} % Clear all header and footer fields
\fancyfoot[R]{\textit{\thepage/\pageref{LastPage}}} % Centered footer with page number
\renewcommand{\headrulewidth}{0pt} % Remove the header line
\usepackage{lastpage}

\newcommand{\blue}[1]{\textcolor{blue}{#1}}
% \usepackage[left=0.6in,top=0.6in,right=0.6in,bottom=0.6in]{geometry} % Document margins

\newcommand{\tab}[1]{\hspace{.2667\textwidth}\rlap{#1}}
\newcommand{\itab}[1]{\hspace{0em}\rlap{#1}}
\name{Wenxue Li} % Your name
% \address{Hong Kong University of Science and Technology} % Your address
%\address{123 Pleasant Lane \\ City, State 12345} % Your secondary addess (optional)
\address{\\ Phone: (+852) 51964362 \\ E-mail: wlicv@connect.ust.hk} % Your phone number and email

\begin{document}

%----------------------------------------------------------------------------------------
%	EDUCATION SECTION
%----------------------------------------------------------------------------------------

\begin{rSection}{Research Interests}
I am broadly interested in datacenter networking and distributed systems, with particular focuses on network protocols, high-speed networking, and networked system for AI.

	
\end{rSection}

\begin{rSection}{Education}
%--copy and paste this region  if you need more--
{\bf Hong Kong University of Science and Technology}{, Hong Kong SAR} \hfill {\em Sept. 2021 - Present}
\\ Ph.D. Candidate in Computer Science and Engineering %\hfill { GPA: 4.0/4.3}
\\ Advisor: Prof. Kai Chen

{\bf Zhejiang University}{, Hangzhou, China} \hfill {\em Sept. 2016 - July 2021} 
\\ Bachelor Degree in Electronic Science and Technology %\hfill {GPA: 3.86/4.0}


%--copy and paste this region  if you need more--

\end{rSection}
%----------------------------------------------------------------------------------------
%	EXPERIENCE SECTION
%----------------------------------------------------------------------------------------
\begin{rSection}{Publications}
* indicates co-first authors 

\textbf{Conference Publications}
%--copy and paste this region  if you need more--

[C7] \textbf{Wenxue Li}, Xiangzhou Liu, Yunxuan Zhang, Zihao Wang, Wei Gu, Tao Qian, Gaoxiong Zeng, Shoushou Ren, Xinyang Huang, Zhenghang Ren, Bowen Liu, Junxue Zhang, Bingyang Liu, Kai Chen. \blue{Revisiting RDMA Reliability for Lossy Fabrics.} In the 39th annual conference of the ACM Special Interest Group on Data Communication (\textbf{SIGCOMM}), Coimbra, Portugal, September 8-11, 2025.

[C6] Bowen Liu, Xinyang Huang, Qijing Li, Zhuobin Huang, Yijun Sun, \textbf{Wenxue Li}, Junxue Zhang, Ping Yin, Kai Chen. \blue{CEIO: A Cache-Efficient Network I/O Architecture for NIC-CPU Data Paths.} In the 39th annual conference of the ACM Special Interest Group on Data Communication (\textbf{SIGCOMM}), Coimbra, Portugal, September 8-11, 2025.

[C5] Jinbin Hu*, \textbf{Wenxue Li*}, Xiangzhou Liu,, Junfeng Wang, Bowen Liu, Ping Yin, Jianxin Wang, Jiawei Huang, Kai Chen. \blue{FLB: Fine-grained Load Balancing for Lossless Datacenter Networks.} In the 2025 USENIX Annual Technical Conference (\textbf{ATC}), Boston, MA, USA, July 2025.

[C4] Zhenghang Ren, Yuxuan Li, Zilong Wang, Xinyang Huang, \textbf{Wenxue Li}, Kaiqiang Xu, Xudong Liao, Yijun Sun, Bowen Liu, Han Tian, Junxue Zhang, Mingfei Wang, Zhizhen Zhong, Guyue Liu, Ying Zhang, Kai Chen. \blue{Enabling Efficient GPU Communication over Multiple NICs with FuseLink.} In the 19th USENIX Symposium on Operating Systems Design and Implementation (\textbf{OSDI}), Boston, MA, USA, July 2025.

[C3] \textbf{Wenxue Li*}, Junyi Zhang*, Yufei Liu, Gaoxiong Zeng, Zilong Wang, Chaoliang Zeng, Pengpeng Zhou, Qiaoling Wang, Kai Chen. \blue{Cepheus: Accelerating Datacenter Applications with High-Performance Roce-Capable Multicast.} In the 30th IEEE International Symposium on High-Performance Computer Architecture (\textbf{HPCA}), Edinburgh, Scotland, March 2024.

[C2] \textbf{Wenxue Li}, Chaoliang Zeng, Jinbin Hu, Kai Chen. \blue{Towards Fine-grained and Practical Flow Control for Datacenter Networks.} In the 31st IEEE International Conference on Network Protocols (\textbf{ICNP}), Reykjavik, Iceland, October 2023.

[C1] Zilong Wang, Layong Luo, Qingsong Ning, Chaoliang Zeng, \textbf{Wenxue Li}, Xinchen Wan, Peng Xie, Tao Feng, Ke Cheng, Xiongfei Geng, Weicheng Ling, Kejia Huo, Pingbo An, Kui Ji, Shideng Zhang, Bin Xu, Ruiqing Feng, Tao Ding, Kai Chen, Chuanxiong Guo. \blue{SRNIC: A Scalable Architecture for RDMA NICs.} In the 20th USENIX Symposium on Networked Systems Design and Implementation (\textbf{NSDI}), Boston, MA, USA, April 2023.

\textbf{Journal Publications}

[J1] \textbf{Wenxue Li}, Chaoliang Zeng, Jinbin Hu, Kai Chen. \blue{FlowSail: Fine-grained and Practical Flow Control for Datacenter Networks.} In IEEE/ACM Transactions on Networking (\textbf{ToN}), 2024

\textbf{Workshop Publications}

[W4] Xiangzhou Liu, \textbf{Wenxue Li}, Kai Chen. \blue{Enabling Packet Spraying over Commodity RNICs with In-Network Support.} In 7th Asia-Pacific Workshop on Networking (\textbf{APNet}), Shanghai, China, Augest 7-8, 2025.

[W3] Yuxuan Li, Zhenghang Ren, \textbf{Wenxue Li}, Xiangzhou Liu, Kai Chen. \blue{Congestion Control for AI Workloads with Message-Level Signaling.} In 7th Asia-Pacific Workshop on Networking (\textbf{APNet}), Shanghai, China, Augest 7-8, 2025.

[W2] \textbf{Wenxue Li}, Xiangzhou Liu, Yuxuan Li, Yilun Jin, Han Tian, Zhizhen Zhong, Guyue Liu, Ying Zhang, Kai Chen. \blue{Understanding Communication Characteristics of Distributed Training.} In 8th Asia-Pacific Workshop on Networking (\textbf{APNet}), Sydney, Australia, August 2024.

[W1] \textbf{Wenxue Li}, Chaoliang Zeng, Jinbin Hu, Kai Chen. \blue{Scaling Switch-driven Flow Control with Aquarius.} In the 7th Asia-Pacific Workshop on Networking (\textbf{APNet}), Hong Kong SAR, June 2023.

%--copy and paste this region  if you need more--
\end{rSection}
%--------------------------------------------------------------------------------
%    PROJECTS
%-----------------------------------------------------------------------------------------------
\begin{rSection}{Experiences}

\textbf{Research Intern, Huawei Hong Kong Research Center}{\hfill June 2024 - Present}\\
Hong Kong SAR

\textbf{Software Engineer Intern, ByteDance}{\hfill July 2020 - June 2021}\\
Hangzhou, China
%	{\small $\bullet$ \textit{RDMA Program Development}: developing RDMA communication programs in virtualized environment.}\\
%	{\small $\bullet$ \textit{Resource Scheduling}: using reinforcement learning (RL) to allocate CPU Cores to different applications in real time. This project achieves better resource utilization and enhances overall performance by 40\% on a test cluster with hundreds of machines.}
	

\end{rSection}


\begin{rSection}{Skills}

{\bf Programming}{\quad C/C++, Python, Golang, P4}
\\{\bf Tools}{\quad NS3, OpenMPI, Pytorch, \LaTeX}
\\{\bf Languages}{\quad Mandarin (native), English (fluent)}
%--copy and paste this region  if you need more--
\end{rSection}
%--------------------------------------------------------------------------------
%    ACTIVITIES
%-----------------------------------------------------------------------------------------------
\begin{rSection}{Academic Services}
\textbf{Journal Reviewer}\\
ACM Transactions on Privacy and Security (TOPS) \hfill {2024}

%--copy and paste this region  if you need more--
\textbf{Teaching Assistant}\\
{\small \textit{COMP3511} Operating System \hfill {2025 Spring}}\\
{\small \textit{COMP4621} Cloud Computing and Big Data Systems \hfill {2022 Spring}}\\
{\small \textit{COMP3511} Operating System \hfill {2022 Fall}}\\
{\small \textit{COMP5621} Computer Networks \hfill {2023 Fall}}

% {\bf Conference Reviewer}\\
% {\small Asia and South Pacific Design Automation Conference (ASP-DAC) \hfill 2023, 2024} \\
% {\small International Conference on Hardware/Software Co-design and System Synthesis (CODES+ISSS) \hfill 2021, 2022}





%{\bf Journal Reviewer}\\
%{\small IEEE Transactions on Computer-Aided Design of Integrated Circuits And System (TCAD)}\\
%{\small IEEE Transactions on Very Large Scale Integration Systems (TVLSI)}
%--copy and paste this region  if you need more--
\end{rSection}


\begin{rSection}{Honors}
HPCA 2024 Travel Grants \hfill{\em 2024}\\
HKUST Research Travel Grants \hfill{\em2023 Fall, 2024 Winter, 2024 Fall}\\
Postgraduate Studentship, HKUST\hfill{\em 2021-2025}\\
First-class Scholarship, ZJU\hfill{\em 2019}\\
Zhejiang Daily - Alibaba New Media Scholarship, ZJU\hfill{\em 2019}\\
Outstanding Student Leader, ZJU\hfill{\em 2018}

\end{rSection}

\end{document}----------------------------

